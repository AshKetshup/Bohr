\chapter{Desenvolvimento e Implementação}
\label{ch::implement}

\section{Introdução}
\label{sec::implement:intro}

A fase de implementação, que se estendeu por cerca de \textbf{ADICIONAR TEXTO}, envolveu a execução paralela de diferentes tarefas pelos dois elementos do grupo. Este Capítulo aborda em particular os seguintes aspetos desta fase do projeto:

\begin{itemize}%[nosep]
	\item Escolhas de implementação (Secção \ref{sec::implement:escolhas}): explica as decisões feitas durante a implementação do código-fonte;
	\item Detalhes de implementação (Secção \ref{sec::implement:detalhes}): explora os detalhes mais importantes e/ou interessantes no código-fonte.
\end{itemize}

Adicionalmente, são descritos o manual de instalação (Secção \ref{sec::implement:instalar}) e %: aborda a compilação da aplicação e a respetiva instalação num \textit{smartphone} com sistema operativo \textit{Android}\texttrademark;
o manual de utilização (Secção \ref{sec::implement:utilizacao}). %: exemplifica o uso da aplicação final na ótica do utilizador.


\section{Escolhas de Implementação}
\label{sec::implement:escolhas}

Text\ldots


\section{Detalhes de Implementação}
\label{sec::implement:detalhes}

Text\ldots



\section{Manual de Instalação}
\label{sec::implement:instalar}

Text\ldots



\section{Manual de Utilização}
\label{sec::implement:utilizacao}

Text\ldots


\section{Conclusões}
\label{chap3:sec:concs}

A exposição dos pontos mais importantes relacionados com a fase de implementação da aplicação \theapp~permitiu ao grupo fazer uma retrospetiva do seu trabalho e perceber quais foram os pontos fortes e os pontos fracos do resultado final. Tal abre a porta para a última fase do projeto, uma fase sem código nem questões técnicas: a reflexão crítica.
