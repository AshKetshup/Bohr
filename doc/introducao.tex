\chapter{Introdução}
\label{chap:intro}

\section{Descrição da proposta}
\label{sec::intro:descricao}

Várias áreas científicas de elevada importância na sociedade moderna dependem do estudo ao nível molecular e atómico dos componentes que sustentam a vida na Terra. Entre elas destacam-se em particular a indústria farmacêutica, a biotecnologia e as áreas afetas como a química orgânica.

Contudo, apesar da existência de vários \textit{softwares} no mercado para a visualização e estudo destas, é de grande interesse na área da \ac{CG} perceber como se pode renderizar estas moléculas de forma a poderem ser observadas virtualmente e estudadas.

Neste sentido, o grupo desafiou-se a desenvolver um visualizador molecular simplificado de ficheiros \ac{PDB} no âmbito da \ac{UC} de \ac{CG} ao invés dos projetos clássicos propostos inicialmente.

A proposta inicial consiste num visualizador molecular que renderize a superfície implícita \textit{pi} \cite{DBLP:journals/corr/abs-1906-06751}. Contudo, devido a contratempos descritos \textit{a posteriori} no presente documento, a implementação final centra-se na renderização da superfície de \textit{van der Walls}.


\section{Constituição do grupo}
\label{sec::intro:grupo}

O presente projeto foi realizado pela equipa constituída pelos elementos listados na Tabela \ref{tab::team}. O trabalho realizado por cada membro é descrito na Secção \ref{sec::reflexao:divisao}.

\begin{table}[!h]
	\centering
	\begin{tabular}{c l}
		\toprule
		\textbf{N\textordmasculine} & \textbf{Nome} \\
		\midrule
		41266 & Diogo Castanheira Simões    \\
		41381 & Igor Cordeiro Bordalo Nunes \\
		\bottomrule
	\end{tabular}
	\caption[Constituição do grupo]{Constituição do grupo.}
	\label{tab::team}
\end{table}


\section{Organização do Documento}
\label{sec::intro:organizacao}
% !POR EXEMPLO!
De modo a refletir o projeto realizado, este relatório encontra-se estruturado em cinco capítulos:

\begin{enumerate}
\item No primeiro capítulo --- \textbf{Introdução} --- é apresentado o projeto, em particular os seus objetivos, a equipa desenvolvedora, a respetiva organização do relatório. % e um breve resumo do Estado da Arte do tema por si abordado.

\item No segundo capítulo --- \textbf{Tecnologias utilizadas} --- são delineadas as tecnologias utilizadas durante o seu desenvolvimento.

\item No terceiro capítulo --- \textbf{Desenvolvimento e Implementação} --- são descritas as escolhas e os detalhes de implementação da aplicação.

\item No quarto capítulo --- \textbf{Reflexão Crítica e Problemas Encontrados} --- são indicados os objetivos alcançados, quais as tarefas realizadas por cada membro do grupo, assim como são expostos os problemas enfrentados e é feita uma reflexão crítica sobre o trabalho.

\item No quinto capítulo --- \textbf{Conclusões e Trabalho Futuro} --- são analisados os conhecimentos adquiridos ao longo do desenvolvimento do projeto e, em contrapartida, o que não se conseguiu alcançar e que poderá ser explorado futuramente.
\end{enumerate}


% \section{Resumo do Estado da Arte}
% \label{sec::intro:estado-arte}
% 
% Text\ldots