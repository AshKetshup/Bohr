\chapter{Reflexão Crítica e Problemas Encontrados}
\label{ch::reflexao}

\section{Introdução}
\label{sec::reflexao:intro}

Não obstante o planeamento feito \textit{a priori}, o projeto \theapp, tal como qualquer outro na área das \ac{TI}, enfrentou alguns contratempos e, devido a problemas de saúde e pessoais severos por parte de um dos membros do grupo, não se revelou possível almejar todas as ambições inicialmente imaginadas. É preciso, pois, refletir sobre o desenvolvimento deste projeto.

Neste Capítulo são, portanto, explorados os seguintes tópicos:

\begin{itemize}
	\item Objetivos propostos vs. alcançados (Secção \ref{sec::reflexao:objetivos}): compara os objetivos inicialmente propostos com aqueles que foram concluídos no projeto final;
	\item Divisão de trabalho pelos elementos do grupo (Secção \ref{sec::reflexao:divisao}): lista as tarefas realizadas por cada elemento da equipa;
	\item Problemas encontrados (Secção \ref{sec::reflexao:problemas}): na sequência da Secção \ref{sec::reflexao:objetivos}, explora os problemas encontrados durante a implementação da aplicação;
	\item Reflexão crítica (Secção \ref{sec::reflexao:critica}): é feita uma \ac{SWOT} em retrospetiva pela equipa acerca do projeto.
\end{itemize}



\section{Objetivos Propostos vs. Alcançados}
\label{sec::reflexao:objetivos}

A Tabela \ref{tab::objetivos} expõe os objetivos propostos inicialmente para o projeto e identifica quais foram alcançados na sua plenitude, quais foram alcançados apenas parcialmente, e quais não tiveram sucesso.

\begin{table}[!htbp]
	\centering
	\begin{tabular}{p{.65\textwidth} >{\centering\let\newline\\\arraybackslash\hspace{0pt}}m{.25\textwidth}}
		\toprule
		{\bfseries Objetivo proposto} & {\bfseries Alcançado?} \\
		\midrule
		\textit{Parsing} de ficheiros \acs{PDB}              & $\bullet$ \\
		Computação das esferas                               & $\bullet$ \\
		Obtenção do raio de cada átomo                       & $\bullet$ \\
		Obtenção da cor CPK de cada átomo                    & $\bullet$ \\
		Renderização de uma esfera                           & $\bullet$ \\
		Renderização de uma molécula                         & $\bullet$ \\
		Rotação de uma molécula                              & $\bullet$ \\
		Movimentação da câmara                               & $\bullet$ \\
		Abertura dinâmica de ficheiros por interface gráfica & $\bullet$ \\
		Renderização da superfície de \textit{van der Walls} & $\bullet$ \\
        Suporte a proteínas                                  & $\circ$   \\
        Renderização da superfície implícita \textit{pi}     & --        \\
        Utilização regular do \textit{Trello}                & --        \\
		\bottomrule
	\end{tabular}
	\caption[Objetivos propostos vs. alcançados]{
		Objetivos propostos e respetiva indicação de sucesso.\\
		\textit{Legenda.} $\bullet$ Alcançado em pleno; $\circ$ Alcançado parcialmente. -- Não alcançado.
	}
	\label{tab::objetivos}
\end{table}


\begin{table}[!htbp]
	\centering
	\begin{tabular}{l c c}
		\toprule
		\textbf{Tarefa} & \textbf{DS} & \textbf{IN} \\
		\midrule
        \textit{Parsing} de ficheiros \acs{PDB}              &  & $\bullet$ \\
        Computação das esferas                               &  & $\bullet$ \\
        Obtenção do raio de cada átomo                       & $\bullet$ &  \\
        Obtenção da cor CPK de cada átomo                    & $\bullet$ &  \\
        Renderização de uma esfera                           &  & $\bullet$ \\
        Renderização de uma molécula                         &  & $\bullet$ \\
        Rotação de uma molécula                              & $\bullet$ &  \\
        Movimentação da câmara                               & $\bullet$ &  \\
        Abertura dinâmica de ficheiros por interface gráfica &  & $\bullet$ \\
        Renderização da superfície de \textit{van der Walls} &  & $\bullet$ \\
		\bottomrule
	\end{tabular}
	\caption[Distribuição de tarefas]{
		Distribuição de tarefas pelos elementos do grupo.\\
		\textit{Legenda.}~%
		$\bullet$ principal responsável; $\circ$ auxiliou.
		DS: Diogo Simões; IN: Igor Nunes.
	}
	\label{tab::divisao-trabalho}
\end{table}


\section{Divisão de Trabalho pelos Elementos do Grupo}
\label{sec::reflexao:divisao}

A fim de agilizar o desenvolvimento do projeto, as tarefas foram distribuídas de forma balanceada pelos dois elementos da equipa (Tabela \ref{tab::divisao-trabalho}). De notar que, apesar da divisão de tarefas, a participação do grupo no seu todo foi essencial nos momentos de entreajuda, esclarecimento de dúvidas e consolidação da aplicação final.

% Reuniões semanais foram realizadas com \textit{deadlines} bem definidas para cada tarefa, tendo sido realizadas reuniões extraordinárias em momentos críticos do desenvolvimento.



\section{Problemas Encontrados}
\label{sec::reflexao:problemas}

A implementação da aplicação levou a que o grupo encontrasse alguns problemas e contratempos, os quais teve de ultrapassar a fim de terminar o projeto. Os problemas mais notáveis são resumidos na Tabela \ref{tab::problemas}, incluindo as soluções encontradas para os ultrapassar.

\begin{table}[!htbp]
	\centering
	\begin{tabular}{p{.36\textwidth} p{.56\textwidth}}
		\toprule
		{\bfseries Problema} & {\bfseries Solução} \\
		\midrule
		\midrule
        \textit{Parsing} dos ficheiros \ac{PDB}. & Estudo do formato segundo descrito em diversos fóruns \textit{online} a fim de determinar as posições exatas dos dados relativo a cada átomo. \\
        \midrule
		Procura do método mais eficiente para renderizar uma esfera. & Utilização da classe \textit{Sphere} disponibilizada no \textit{OpenGL Shading Language Cookbook, 2nd Edition} \cite{vbosphere} a fim de evitar a importação de objetos externos. \\
		\midrule
		Determinação da superfície implícita \textit{pi}. & Sem solução encontrada. É necessário mais estudo. \\
		\midrule
		Renderização da superfície implícita \textit{pi} e dificuldade do algoritmo \textit{marching cubes}. & Adoção do algoritmo \textit{marching triangles} com recurso ao código \textit{open source} disponível em \cite{marchingtriangles}. Solução não presente na aplicação final devido ao problema anterior não solucionado. \\
		\bottomrule
	\end{tabular}
	\caption[Problemas encontrados e respetivas soluções]{Problemas encontrados durante o desenvolvimento da aplicação e respetivas soluções adotadas.}
	\label{tab::problemas}
\end{table}



\section{Reflexão Crítica}
\label{sec::reflexao:critica}

Tendo por objetivo expor a reflexão da \groupname~face ao trabalho enveredado no desenvolvimento da aplicação \theapp, propõe-se efetivá-la com uma análise \ac{SWOT}.

\subsection{Pontos Fortes}
\label{ssec::reflexao:critica:fortes}

\begin{enumerate}[nosep]
	\item A aplicação permite abrir uma nova molécula a qualquer momento.
    \item É utilizada interface gráfica do sistema operativo para interagir com o utilizador.
    \item As esferas são calculadas dinamicamente pela aplicação sem recurso a objetos externos.
    \item O renderizador tem em conta a cor de cada átomo e o respetivo raio de \textit{van der Walls}, mostrando assim a superfície de \textit{van der Walls}.
\end{enumerate}


\subsection{Pontos Fracos}
\label{ssec::reflexao:critica:fracos}

\begin{enumerate}[nosep]
	\item O \textit{parser} \ac{PDB} não contempla ficheiros relativos a proteínas.
    \item O mesmo código-fonte compilado em diferentes sistemas operativos não garante exatamente o mesmo comportamento em relação à câmara e ao teclado.
    \item Não é calculada a superfície implícita \textit{pi}, não sendo então possível renderizar esta.
    \item Só é renderizado um tipo de superfície molecular.
\end{enumerate}


\subsection{Ameaças}
\label{ssec::reflexao:critica:ameacas}

\begin{enumerate}[nosep]
	\item O \textit{software} não contempla a possibilidade de fechar a molécula atualmente aberta nem se mostrar várias moléculas lado-a-lado.
    \item O uso de interface gráfica não tem um comportamento garantido em \textit{cross-platform}.
    \item Para moléculas de grandes dimensões, a aplicação poderá facilmente tornar-se exigente em termos de recursos de memória.
    \item Já existem variados \textit{softwares} de visualização molecular, apesar da pertinência deste projeto no âmbito da \ac{UC} de \ac{CG}.
\end{enumerate}


\subsection{Oportunidades}
\label{ssec::reflexao:critica:oportunidades}

\begin{enumerate}[nosep]
	\item Adicionar suporte para aminoácidos.
    \item Adaptar com \textit{flags} do pré-processador do C++ o código-fonte a fim de ultrapassar as limitações em cada sistema operativo.
    \item Determinar a superfície implícita \textit{pi} a fim de dar uso ao algoritmo \textit{marching triangles} estudado para este projeto.
    \item Renderizar mais superfícies, tendo em conta os vários raios armazenados na tabela periódica interna da aplicação.
\end{enumerate}



\section{Notas finais}
\label{sec::reflexao:notas}

Antes de terminar a reflexão crítica, é pertinente levar em conta os problemas pessoais e de saúde enfrentados por cada um dos membros do grupo que acabou por levar a um significativo atraso da conclusão do projeto face à data prevista.

O grupo reconhece, portanto, este mesmo atraso, o qual culminou no não cumprimento total dos objetivos inicialmente propostos para este projeto.

Não obstante, o facto de esta aplicação ter por base op cálculo e a renderização dinâmica de todos os seus elementos, ao invés de uma abordagem \textit{hard-coded}, levantou novos desafios até então não encontrados durante as aulas práticas da \ac{UC} de \ac{CG}.

Consideramos, portanto, que, apesar do lamentável atraso, o resultado final obtido é satisfatório.


\section{Conclusões}
\label{sec::reflexao:conclusao}

Esta fase de reflexão permitiu analisar o trabalho enveredado ao longo das semanas de planeamento, execução e teste. Com esta análise, a equipa pôde tirar conclusões não só sobre o seu desempenho, mas também acerca das tecnologias utilizadas, as quais serão expostas no Capítulo seguinte.
