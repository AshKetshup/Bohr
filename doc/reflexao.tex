\chapter{Reflexão Crítica e Problemas Encontrados}
\label{ch::reflexao}

\section{Introdução}
\label{sec::reflexao:intro}

Não obstante o bom planeamento feito \textit{a priori}, o projeto \appname, tal como qualquer outro na área das \ac{TI}, enfrentou alguns contratempos e, devido a problemas de saúde e pessoais severos por parte de um dos membros do grupo, não se revelou possível almejar todas as ambições inicialmente imaginadas. É preciso, pois, refletir sobre o desenvolvimento deste projeto.

Neste Capítulo são, portanto, explorados os seguintes tópicos:

\begin{itemize}
	\item Objetivos propostos vs. alcançados (Secção \ref{sec::reflexao:objetivos}): compara os objetivos inicialmente propostos com aqueles que foram concluídos no projeto final;
	\item Divisão de trabalho pelos elementos do grupo (Secção \ref{sec::reflexao:divisao}): lista as tarefas realizadas por cada elemento da equipa;
	\item Problemas encontrados (Secção \ref{sec::reflexao:problemas}): na sequência da Secção \ref{sec::reflexao:objetivos}, explora os problemas encontrados durante a implementação da aplicação;
	\item Reflexão crítica (Secção \ref{sec::reflexao:critica}): é feita uma \ac{SWOT} em retrospetiva pela equipa acerca do projeto.
\end{itemize}



\section{Objetivos Propostos vs. Alcançados}
\label{sec::reflexao:objetivos}

A Tabela \ref{tab::objetivos} expõe os objetivos propostos inicialmente para o projeto e identifica quais foram alcançados na sua plenitude, quais foram alcançados apenas parcialmente, e quais não tiveram sucesso. De notar que foram foram incluídos 3 idiomas: Português de Portugal (\textit{pt-pt}), Inglês Britânico (\textit{en-uk}) e Francês (\textit{fr}).

\begin{table}[!htbp]
	\centering
	\begin{tabular}{p{.65\textwidth} >{\centering\let\newline\\\arraybackslash\hspace{0pt}}m{.25\textwidth}}
		\toprule
		{\bfseries Objetivo proposto} & {\bfseries Alcançado?} \\
		\midrule
		 & $\bullet$ \\
		 & $\bullet$ \\
		 & $\bullet$ \\
		 & $\bullet$ \\
		 & $\bullet$ \\
		 & $\bullet$ \\
		 & $\bullet$ \\
		 & $\circ$   \\
		 & $\circ$   \\
		 & --        \\
		\bottomrule
	\end{tabular}
	\caption[Objetivos propostos vs. alcançados]{
		Objetivos propostos e respetiva indicação de sucesso.\\
		\textit{Legenda.} $\bullet$ Alcançado em pleno; $\circ$ Alcançado parcialmente. -- Não alcançado.
	}
	\label{tab::objetivos}
\end{table}


\begin{table}[!htbp]
	\centering
	\begin{tabular}{l c c}
		\toprule
		\textbf{Tarefa} & \textbf{DS} & \textbf{IN} \\
		\midrule
		 &             &             \\
		 &             &             \\
		 &             &             \\
		 &             &             \\
		 &             & $\bullet$   \\
		 &             & $\bullet$   \\
		 & $\bullet$   &             \\
		 & $\bullet$   &             \\
		 &             &             \\
		 &             &             \\
		 &             & $\bullet$   \\
		 & $\bullet$   & $\circ$     \\
		 &             &             \\
		 &             &             \\
		 &             &             \\
		 &             &             \\
		\bottomrule
	\end{tabular}
	\caption[Distribuição de tarefas]{
		Distribuição de tarefas pelos elementos do grupo.\\
		\textit{Legenda.}~%
		$\bullet$ principal responsável; $\circ$ auxiliou.
		DS: Diogo Simões; IN: Igor Nunes.
	}
	\label{tab::divisao-trabalho}
\end{table}


\section{Divisão de Trabalho pelos Elementos do Grupo}
\label{sec::reflexao:divisao}

A fim de agilizar o desenvolvimento do projeto, as tarefas foram distribuídas de forma balanceada pelos dois elementos da equipa a fim de jogar com os respetivos pontos fortes (Tabela \ref{tab::divisao-trabalho}). De notar que, apesar da divisão de tarefas, a participação do grupo no seu todo foi essencial nos momentos de entreajuda, esclarecimento de dúvidas e consolidação da aplicação final.

% Reuniões semanais foram realizadas com \textit{deadlines} bem definidas para cada tarefa, tendo sido realizadas reuniões extraordinárias em momentos críticos do desenvolvimento.



\section{Problemas Encontrados}
\label{sec::reflexao:problemas}

A implementação da aplicação levou a que o grupo encontrasse alguns problemas e contratempos, os quais teve de ultrapassar a fim de terminar o projeto. Os problemas mais notáveis são resumidos na Tabela \ref{tab::problemas}, incluindo as soluções encontradas para os ultrapassar.

\begin{table}[!htbp]
	\centering
	\begin{tabular}{p{.56\textwidth} p{.36\textwidth}}
		\toprule
		{\bfseries Problema} & {\bfseries Solução} \\
		\midrule
		\midrule
		 &  \\
		\midrule
		 &  \\
		\midrule
		 &  \\
		\midrule
		 &  \\
		\midrule
		 &  \\
		\bottomrule
	\end{tabular}
	\caption[Problemas encontrados e respetivas soluções]{Problemas encontrados durante o desenvolvimento da aplicação e respetivas soluções adotadas.}
	\label{tab::problemas}
\end{table}



\section{Reflexão Crítica}
\label{sec::reflexao:critica}

Tendo por objetivo expor a reflexão da \groupname~face ao trabalho enveredado no desenvolvimento da aplicação \theapp, propõe-se efetivá-la com uma análise \ac{SWOT}.

\subsection{Pontos Fortes}
\label{ssec::reflexao:critica:fortes}

\begin{enumerate}[nosep]
	\item Text\ldots
\end{enumerate}


\subsection{Pontos Fracos}
\label{ssec::reflexao:critica:fracos}

\begin{enumerate}[nosep]
	\item Text\ldots
\end{enumerate}


\subsection{Ameaças}
\label{ssec::reflexao:critica:ameacas}

\begin{enumerate}[nosep]
	\item Text\ldots
\end{enumerate}


\subsection{Oportunidades}
\label{ssec::reflexao:critica:oportunidades}

\begin{enumerate}[nosep]
	\item Text\ldots
\end{enumerate}



\section{Conclusões}
\label{sec::reflexao:conclusao}

Esta fase de reflexão permitiu analisar o trabalho enveredado ao longo das semanas de planeamento, execução e teste. Com esta análise, a equipa pôde tirar conclusões não só sobre o seu desempenho, mas também acerca das tecnologias utilizadas, as quais serão expostas no Capítulo seguinte.
