\chapter{Tecnologias utilizadas}
\label{ch::tecno}


\section{Introdução}
\label{sec::tecno:intro}

Antes da implementação propriamente dita da aplicação, é necessário estabelecer quais as ferramentas e as tecnologias utilizadas a fim de alcançar os objetivos propostos para o projeto.

Este Capítulo aborda este tópico, % to be continued...


\section{Ferramentas e tecnologias utilizadas}
\label{sec::tecno:tecnologia}

As ferramentas utilizadas no âmbito da realização do projeto, sumariadas na Tabela \ref{tab::ferramentas}, visam três componentes essenciais na sua gestão:
\begin{inparaenum}[1)]
	\item aplicação \opengl,
	\item relatório, e
	\item controlo de versões.
\end{inparaenum}

Text\ldots


\begin{table}[!htbp]
	\centering
	\begin{tabular}{p{1cm} l l}
		\toprule
		& {\bfseries \textit{Software} / Tecnologia} & {\bfseries Versão} \\
		\midrule
		\multicolumn{3}{l}{\bfseries Aplicação \opengl} \\
		& \textit{Android Studio} & 4.1.1 \\
		&  &  \\
		&  &  \\
		&  &  \\
		\midrule
		\multicolumn{3}{l}{\bfseries Relatório} \\
		& Xe\TeX & 3.14159265-2.6-0.999991 \\
		& \textit{TeXstudio}\textsuperscript{\textcopyright} & 3.0.1 \\
		\midrule
		\multicolumn{3}{l}{\bfseries Controlo de versões} \\
		& \textit{git} & 2.17.1 \\
		& \textit{GitKraken} & 7.4.1  \\
		\bottomrule
	\end{tabular}
	\caption[Ferramentas utilizadas]{Ferramentas e tecnologias utilizadas, organizadas por categoria.}
	\label{tab::ferramentas}
\end{table}



\section{Conclusões}
\label{sec::tecno:conclusao}

Text\ldots
