\chapter{Tecnologias utilizadas}
\label{ch::tecno}


\section{Introdução}
\label{sec::tecno:intro}

Antes da implementação propriamente dita da aplicação, é necessário estabelecer quais as ferramentas e as tecnologias utilizadas a fim de alcançar os objetivos propostos para o projeto. Este Capítulo aborda, portanto, este tópico.


\section{Ferramentas e tecnologias utilizadas}
\label{sec::tecno:tecnologia}

As ferramentas utilizadas no âmbito da realização do projeto, sumariadas na Tabela \ref{tab::ferramentas}, visam três componentes essenciais na sua gestão:
\begin{inparaenum}[1)]
	\item aplicação \opengl,
	\item relatório, e
	\item controlo de versões.
\end{inparaenum}

Segue-se uma descrição do uso de cada uma das tecnologias:

\begin{itemize}
    \item \textbf{\opengl~\cite{opengl}}: \ac{API} multi-plataforma e com suporte a múltiplas linguagens de programação para a renderização de gráficos vetoriais 2D e 3D com recurso à placa gráfica;
    
    \item \textbf{GLFW~\cite{glfw}}: \ac{API} simplificada para o \opengl, igualmente multi-plataforma, permitindo a gestão de janelas, contextos, superfícies e comandos (rato, teclado e \textit{joystick});
    
    \item \textbf{GLM~\cite{glm}}: biblioteca matemática baseada na linguagem dos \textit{shaders} do \opengl, \ac{GLSL}.
    
    \item \textbf{\ac{GLAD}~\cite{glad,glad-webservice}}: gerador automático de \textit{loaders} para \opengl;
    
    \item \textbf{\textit{FreeType}~\cite{freetype}}: biblioteca de desenvolvimento dedicada à renderização de fontes em \textit{bitmaps} utilizáveis, por exemplo, pelo \opengl.
\end{itemize}

\begin{table}[!htbp]
    \centering
    \begin{tabular}{p{1cm} l l}
        \toprule
        & {\bfseries \textit{Software} / Tecnologia} & {\bfseries Versão} \\
        \midrule
        \multicolumn{3}{l}{\bfseries Aplicação \opengl} \\
        & \opengl           & 4.6 \\
        & GLFW              & 3.3.2 \\
        & GLAD              & 0.1.34 \\
        & \acs{GLM}         & 0.9.9.8 \\
        & \textit{FreeType} & 2.10.4 \\
        \midrule
        \multicolumn{3}{l}{\bfseries Relatório} \\
        & Xe\TeX & 3.14159265-2.6-0.999991 \\
        & \textit{TeXstudio}\textsuperscript{\textcopyright} & 3.0.1 \\
        \midrule
        \multicolumn{3}{l}{\bfseries Controlo de versões} \\
        & \textit{git} & 2.17.1 \\
        & \textit{GitKraken} & 7.4.1  \\
        \bottomrule
    \end{tabular}
    \caption[Ferramentas utilizadas]{Ferramentas e tecnologias utilizadas, organizadas por categoria.}
    \label{tab::ferramentas}
\end{table}


\section{Código \textit{open source}}

Foi utilizado código \textit{open source}, adaptado para o projeto sob as respetivas licenças, para as seguintes funcionalidades:

\begin{itemize}
    \item \textbf{\itshape Sphere}~\cite{glsl-cookbook-github,vbosphere}: obtido dos códigos de exemplo do \textit{OpenGL Shading Language Cookbook, 2nd Edition} para renderizar esferas sem recurso a ficheiros \verb*|*.obj| criados externamente;
    
    \item \textbf{\itshape osdialog}~\cite{osdialog}: biblioteca multi-plataforma para acesso facilitado às caixas de diálogo do sistema operativo, com o fim de permitir abrir qualquer ficheiro \ac{PDB} dinamicamente a qualquer momento e mostrar mensagens de erro ao utilizador.
\end{itemize}

O projeto \theapp~será por sua vez disponibilizado sob a licença \textit{GNU General Public License, version 3}~\cite{gnugpl3}, após apresentação e caso seja autorizado pelo Professor regente.


\section{Conclusões}
\label{sec::tecno:conclusao}

Após delineadas as ferramentas e tecnologias a utilizar, segue-se a fase de desenvolvimento do projeto, descrito no Capítulo seguinte. A utilização de código \textit{open source} e de ferramentas bem conhecidas e testadas pela comunidade mundial permitirá, em princípio, uma implementação eficiente.
